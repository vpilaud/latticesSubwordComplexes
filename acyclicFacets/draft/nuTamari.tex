\subsection{The acyclic property for $\nu$-Tamari lattices} 
\label{subsec:nuTamari} 

We close this section with a theorem that relates our results to the $\nu$-Tamari lattices introduced by L.-F.~Préville-Ratelle and X.~Viennot in~\cite{PrevilleRatelleViennot}. These posets are indexed by a lattice path $\nu$ consisting of a finite number of north~(N) and east~(E) steps, and coincide with the classical Tamari lattices when $\nu=(NE)^n$.

In~\cite{CeballosPadrolSarmiento}, it was shown that the $\nu$-Tamari lattice can be obtained as the increasing flip poset on pipe dreams~$\pipeDreams(\omega_\nu)$ for an explicit permutation $\omega_\nu$ associated to $\nu$. The permutations of the form~$\omega_\nu$ can be easily characterized as follows. We refer to~\cite{CeballosPadrolSarmiento} for details. 

The \defn{Rothe diagram} of a permutation~$\omega$ is the set $\set{(\omega(j),i)}{i<j,\ \omega(i)>\omega(j)}$ in matrix notation.
A permutation $\omega$ is called \defn{dominant} if its Rothe diagram is the Ferrer's diagram of a partition located at the top left corner $(1,1)$. Equivalently, a permutation~$\omega$ is dominant if and only if it is 132-avoiding. 
For a permutation~$\omega = \omega_1 \dots \omega_n \in \fS_n$, we denote by~$0\omega$ the permutation~$0 \omega_1 \dots \omega_n$ of~$\{0, 1, \dots, n\}$, and we consider here pipe dreams with pipes indexed by~$\{0, 1, \dots, n\}$. 


From \cref{exm:Tamari1}, we see that this property holds when $\omega=n\dots21$ is the reverse permutation. 
In this case, the increasing flip poset on~$\pipeDreams(0 \omega)$ is the classical Tamari lattice. Removing vertex $0$ from the contact graph of a pipe dream~$ P\in \pipeDreams(0 \omega)$ returns the binary tree corresponding to $P$ (with the leaves removed and the internal nodes labeled from $1$ to $n$).
The edges of the binary tree are oriented going away from the root of the tree.  
See \cref{fig:bijection}. 

Given a pipe dream $P$, We write $i\to j$ whenever $i$ has a contact with $j$,
when $P$ has a cycle ${\bf C}=(i_1, i_2, \ldots, i_m)$ where $i_1 \to i_2\to\cdots\to i_m\to i_1$, we can always assume that $i_1=\min\{i_1,\ldots,i_m\}$.
To describe the location of crossing and contacts (elbows), we use the coordinate $(c,r)$ where $r$ is the row and $c$ is the column.

\begin{lemma}\label{lem:not_dominant}
For a pipe dream $P$ with permutation $0\omega$ and pipes $i<j$ crossing. If $i$ has an elbow southwest of the crossing and $j$ has an elbow northeast of the crossing, then $\omega$ is not dominant.
\end{lemma}

\begin{proof} Let $(c,r)$ be the position of the crossing. Since $i$ has an elbow southwest of $(c,r)$ we have $r\le i-1$.
Since $j$ has an elbow northeast of the crossing, there are at most $i-2$ pipes that cross north of $(c,r)$ in column $c$. 
Hence, at least $2$ pipes $<i$ reach the Northern border before column $c$.
Excluding $0$, there is at least a pipe $0<h$ such that $h<i<j$ and $\omega^{-1}(h)< \omega^{-1}(j)<\omega^{-1}(i)$.
This is a $132$ pattern in $\omega$, therefor $\omega$ is not dominant. \end{proof}


\nantel{modify intro to reflect this and must thanks Lucas Gagnon for helpful suggestions that shorten our proof}
\begin{theorem}
\label{prob:nuAcyclicProperty}
All pipe dreams with $0 \omega$ are acyclic if and only if $\omega$ is a dominant permutation.
\end{theorem}

\begin{proof} For the forward implication, assume that $\omega$ is not dominant. 
Let $P_0$ be the greedy pipe dream of $0\omega$. It has no crossing in row 0, and cannot be a partition starting at row 1.
There must be a row $r>1$ with more crossing than in row $r-1$. 
Considering the rightmost column $c$ of $P$ with a crossing in row $r$,  the pipes $r<k$ are crossing at $(c,r)$, $\omega^{-1}(k)<\omega^{-1}(r)$ and are in contact in position $(c+1,r-1)$.
Let $Q$ be the pipe dream obtained from $P_0$ by the flip of the crossing at $(c,r)$ with the contact at $(c+1,r-1)$.
The contact of $Q$ at $(c,r)$ gives $r\contactLess{Q} k$. In row $0$ we only have contacts, we have $\omega^{-1}(k)<\omega^{-1}(r)$ and Lemma~\ref{lem:rectangle} gives  that 
$k \contactLess{Q} r$, hence $Q$ is not acyclic.

For the converse, assume we have a pipe dream $P$ that is not acyclic. The contact graph must contain a cycle ${\bf C}=(i, j_1, \ldots, j_m)$ with $i<j_k$ for all $1\le k\le m$. 
The contact $i \to j_1$ is southeast of $i$ and the contact  $j_m\to i$ northwest of $i$. 
Let $j_k$ be the first time where $j_{k-1}\to j_k$ is a contact southeast of $i$ and $j_k \to j_{k+1}$\footnote{with the convention that $j_{m+1}=i$} is a contact northwest of $i$.
This implies that the pipe $j=j_k$ cross $i$ in some position $(c,r)$. Since $i<j$, the pipe $j$ is the vertical pipe of the crossing. 
Remark that all contact $j_{k'}\to j_{k'+1}$ are southeast of $i$ for $1\le k' <k$.

Consider first the case $m=1$. The contact $i\to j$ implies there is an elbow  of $i$ southwest of the crossing and  $j\to i$ implies there is an elbow of $j$  northeast of the crossing. 
Lemma~\ref{lem:not_dominant} implies  $\omega$ is not dominant.
Assume now that $m>1$. If $i$ has an elbow southwest of the crossing, then Lemma~\ref{lem:not_dominant} implies  $\omega$ is not dominant.
Thus, we may as well ssume that $i$ has no elbow west of $(c,r)$, and therefor $r=i$. The contact $i\to j_1$ must be southeast of both $i$ and $j$ and the  $j_{k-1}\to j_k=j$ is northwest of $j$.
Let  $j'=j_{k'}$ for $1\le k' <k$ be the first pipe such that the $j_{k'-1}\to j'$ is southeast of $j$ and $j'\to j_{k'+1}$ is northwest of $j$, and both are southeast of $i$. 
Hence $j$ and $j'$ are crossing at some position $(c',r')$ and the pipe $j'$ remain southeast of $i$ between these two contacts.
Remark that now all contacts $j_{k''}\to j_{k''+1}$ are southeast of both pipes  $i$ and $j$ for $1\le k'' <k' < k$.


if $j'<j$, then the contact $j'\to j_{k'+1}$ is southwest of the crossing at $(c',r')$ and  $r'\le j'-1$ and $j$ has a contact north of $(c,r)$ which is north of $(c',r')$. We can apply Lemma~\ref{lem:not_dominant} to 
 $j'<j$ and $\omega$ is not dominant. if $j'>j$, and $j$ has no elbow southwest of $(c',r')$, then the pipe $j'$ must cross $i$ in row $i$ due west of $(c,i)$.
We now repeat the argument above: we have the contact $i\to j_1$ must be southeast of $i$, $j$ and $j'$,
and the contact $j_{k'-1}\to j_{k'}=j'$ is northwest of $j'$.
Let  $j''=j_{k''}$ for $1\le k''<k' <k$ be the first pipe such that the $j_{k''-1}\to j''$ is southeast of $j'$ and $j''\to j_{k''+1}$ is northwest of $j'$, and both are southeast of $i$ and $j$. 
Hence $j'$ and $j''$ are crossing at some position $(c'',r'')$ and the pipe $j''$ remain southeast of $i$ and $j$ between these two contacts.
Remark that now all contacts $j_{k^{(3)}}\to j_{k^{(3)}+1}$ are southeast of both pipes  $i$, $j$ and $j'$ for $1\le k^{(3)}<k'' <k' < k$.

The process above cannot continue at infinitum since $k-1$ is finite. At one point we will find $j^{(\ell)}<j^{(\ell-1)}$ where Lemma~\ref{lem:not_dominant} applies and conclude that $\omega$ is not dominant
in all cases. 
\end{proof}

Applying our Theorem~\ref{thm:pipeDreamQuotient}, we get the following consequence.

\begin{corollary}
If the statement of Open Problem~\ref{prob:nuAcyclicProperty} holds, then the $\nu$-Tamari lattice is a lattice quotient of the interval~$[e,\omega_\nu]$.
\end{corollary}  
